@comment -*-texinfo-*-
@comment This file was generated by doc2tex.pl from start.doc
@comment DO NOT EDIT DIRECTLY, BUT EDIT start.doc INSTEAD
@comment Id: start.tex,v 1.1 2003/08/08 14:27:06 pertusus Exp $
@comment this file contains the "Introduction" chapter.
@c * wichmann: + added changes by GMG.

@c The following directives are necessary for proper compilation
@c with emacs (C-c C-e C-r).  Please keep it as it is.  Since it
@c is wrapped in `@ignore' and `@end ignore' it does not harm `tex' or
@c `makeinfo' but is a great help in editing this file (emacs
@c ignores the `@ignore').
@ignore
%**start
\input texinfo.tex
@setfilename start.hlp
@node Top, Introduction
@menu
* Introduction::
@end menu
@node Introduction, General concepts, Preface, Top
@chapter Introduction
%**end
@end ignore


@ifset singularmanual
@menu
* Background::
* How to use this manual::
* Getting started::
@end menu
@end ifset
@ifclear singularmanual
@menu
* Background::
* How to use this tutorial::
* Getting started::
@end menu
@end ifclear

@c ------------------------------------------------------------------
@ifset singularmanual
@node Background, How to use this manual, Introduction, Introduction
@end ifset
@ifclear singularmanual
@node Background, How to use this tutorial, Introduction, Introduction
@end ifclear
@section Background
@cindex Background

@sc{Singular} is a Computer Algebra system for polynomial
computations with emphasis on the special needs of commutative
algebra, algebraic geometry, and singularity theory.

@sc{Singular}'s main computational objects are ideals and
modules over a large variety of baserings. The baserings are polynomial
rings or localizations thereof over a field (e.g., finite fields, the
rationals, floats, algebraic extensions, transcendental extensions) or
quotient rings with respect to an ideal.

@sc{Singular} features one of the fastest and most general
implementations of various algorithms for computing Groebner
resp.@: standard bases. The implementation includes Buchberger's algorithm
(if the ordering is a well ordering) and Mora's algorithm (if the
ordering is a tangent cone ordering) as special cases.  Furthermore, it
provides polynomial factorizations, resultant, characteristic set and
gcd computations, syzygy and free-resolution computations, and many more
related functionalities.

Based on an easy-to-use interactive shell and a C-like programming
language, @sc{Singular}'s internal functionality is augmented and
user-extendible by libraries written in the @sc{Singular} programming
language. A general and efficient implementation of communication links
allows @sc{Singular} to make its functionality available to other
programs.

@sc{Singular}'s development started in 1984 with an implementation of
Mora's Tangent Cone algorithm in Modula-2 on an Atari computer (K.P.
Neuendorf, G. Pfister,
@ifinfo
H.@: Schoenemann; Humboldt-Universitaet
@end ifinfo
@tex
H.\ Sch\"onemann; Humboldt-Universit\"at
@end tex
 zu Berlin).  The need for a new system arose from the investigation of
mathematical problems coming from singularity theory which none of the
existing systems was able to compute.

In the early 1990s @sc{Singular}'s "home-town" moved to
Kaiserslautern, a general standard basis algorithm was implemented in C
and @sc{Singular} was ported to Unix, MS-DOS, Windows NT, and MacOS.

Continuous extensions (like polynomial factorization, gcd computations,
links) and refinements led in 1997 to the release of @sc{Singular}
version 1.0 and in 1998 to the release of version 1.2
(much faster standard and Groebner bases computations based on Hilbert series
and on improved implementations of the algorithms,
libraries for primary decomposition, ring normalization, etc.)

For the highlights of the new @sc{Singular} version @value{VERSION} see @ref{News and changes}.

@c Thus, we hope to offer a useful system
@c for dealing with local and global computational aspects
@c of systems of polynomial equations.

@c ------------------------------------------------------------------
@ifset singularmanual
@node How to use this manual, Getting started, Background, Introduction
@section How to use this manual
@cindex How to use this manual
@end ifset
@ifclear singularmanual
@node How to use this tutorial, Getting started, Background, Introduction
@section How to use this tutorial
@cindex How to use this tutorial
@end ifclear

@ifset singularmanual
@subsubheading For the impatient user
@end ifset
In @ref{Getting started}, some simple examples explain how to use
@sc{Singular}  in a step-by-step manner.

@ref{Examples} should come next for real learning-by-doing or to quickly
solve some given mathematical problems without dwelling too deeply into
@sc{Singular}.
@ifset singularmanual
This chapter contains a lot of real-life examples and
detailed instructions and explanations on how to solve mathematical
problems using @sc{Singular}.
@end ifset

@c ------------------------------------------------------------------------
@ifset singularmanual
@subsubheading For the systematic user
In @ref{General concepts}, all basic concepts which are important to use
and to understand @sc{Singular} are developed.  But even for users
preferring the systematic approach it will be helpful to have a look at
the examples in @ref{Getting started}, every now and then. The topics in
the chapter are organized more or less in the order the novice user has
to deal with them.

@itemize @bullet
@item
In @ref{Interactive use}, and its subsections there are some words on
entering and exiting @sc{Singular}, followed by a number of other
aspects concerning the interactive user-interface.

@item
To do anything more than trivial integer computations, one needs to
define a basering in @sc{Singular}.  This is explained in detail in
@ref{Rings and orderings}.

@item
An overview of the algorithms implemented in the kernel of @sc{Singular}
is given in @ref{Implemented algorithms}.

@item
In @ref{The SINGULAR language}, language specific concepts are
introduced such as the notions of names and objects, data types and
conversion between them, etc.

@item
In @ref{Input and output}, @sc{Singular}'s mechanisms to store and
retrieve data are discussed.

@item
The more complex concepts of procedures and libraries as
well as tools to debug them are considered in the following sections:
@ref{Procedures}, @ref{Libraries}, and @ref{Debugging tools}.

@end itemize

@ref{Data types}, is a complete treatment for @sc{Singular}'s data types
where each section corresponds to one data type, alphabetically sorted.
For each data type, its purpose is explained, the syntax of its
declaration is given, and related operations and functions are
listed. Examples illustrate its usage.

@ref{Functions and system variables}, is an alphabetically ordered
reference list of all of @sc{Singular}'s functions, control structures,
and system variables.  Each entry includes a description of the syntax
and semantics of the item being explained as well as one or more
examples on how to use it.

@subsubheading Miscellaneous
@ref{Tricks and pitfalls}, is a loose collection of limitations and
features which may be unexpected by those who expect the
@sc{Singular} language to be an exact copy of the C programming language or of
some Computer Algebra system's languages.  But some mathematical hints are
collected there, as well.

@ref{Mathematical background}, introduces some of the mathematical
notions and definitions used throughout this manual.  For example, if in
doubt what exactly @sc{Singular} means by a ``negative degree reverse
lexicographical ordering'' one should refer to this chapter.

@ref{SINGULAR libraries}, lists the
libraries which come with @sc{Singular} and the functions contained in
them, respectively.
@end ifset
@c ------------------------------------------------------------------------

@subsubheading Typographical conventions
Throughout this manual, the following typographical conventions are
adopted:

@itemize @bullet
@item
text in @code{typewriter} denotes @sc{Singular} input and output as well
as reserved names:

@itemize @asis
@item The basering can be set using the command @code{setring}.
@end itemize

@item
the arrow @expansion{} denotes @sc{Singular} output:

@itemize @asis
@item @code{poly p=x+y+z;}
@item @code{p*p;}
@item @code{@expansion{} x2+2xy+y2+2xz+2yz+z2}
@end itemize

@item
square brackets are used to denote parts of syntax descriptions which
are optional:

@itemize @asis
[optional_text] required_text
@end itemize

@item
keys are denoted using typewriter, for example:

@itemize @asis
@item @code{N} (press the key @code{N} to get to the next node in help
mode)
@item @code{RETURN} (press @code{RETURN} to finish an input line)
@item @code{CTRL-P} (press control key together with the key @code{P} to
get the previous input line)
@end itemize

@end itemize

@c ------------------------------------------------------------------
@ifset singularmanual
@node Getting started,  , How to use this manual, Introduction
@section Getting started
@end ifset
@ifclear singularmanual
@node Getting started,  , How to use this tutorial, Introduction
@chapter Getting started
@end ifclear
@cindex Getting started

@sc{Singular} is a special purpose system for polynomial
computations. Hence, most of the powerful computations in @sc{Singular}
require the prior definition of a ring. Most important rings are
polynomial rings over a field, localizations hereof, or quotient rings of
such rings modulo an ideal. However, some simple computations with
integers (machine integers of limited size) and manipulations of strings
are available without a ring.

@menu
* First steps::
* Rings and standard bases::
* Procedures and libraries::
* Change of rings::
* Modules and their annihilator::
* Resolution::
@end menu

@c ------------------------------------------------------------------
@node First steps, Rings and standard bases, Getting started, Getting started
@ifset singularmanual
@subsection First steps
@end ifset
@ifclear singularmanual
@section First steps
@end ifclear
@cindex First steps

Once @sc{Singular} is started, it awaits an input after the prompt
@code{>}.  Every statement has to be terminated by @code{;} .

@smallexample
37+5;
@expansion{} 42
@end smallexample

All objects have a type, e.g., integer variables are defined by
the word @code{int}. An assignment is done by the symbol @code{=} .

@smallexample
int k = 2;
@end smallexample

@noindent Test for equality resp.@: inequality is done using @code{==}
resp.@: @code{!=} (or @code{<>}), where @code{0} represents the boolean
value FALSE, any other value represents TRUE.

@smallexample
k == 2;
@expansion{} 1
k != 2;
@expansion{} 0
@end smallexample

@noindent The value of an object is displayed by simply typing its name.

@smallexample
k;
@expansion{} 2
@end smallexample

@noindent On the other hand the output is suppressed if an assignment
is made.

@smallexample
int j;
j = k+1;
@end smallexample

@noindent The last displayed (!) result is always available
with the special symbol @code{_} .

@smallexample
2*_;   // the value from k displayed above
@expansion{} 4
@end smallexample

Text starting with @code{//} denotes a comment and is ignored in
calculations, as seen in the previous example. Furthermore @sc{Singular}
maintains a history of the previous lines of input, which may be accessed by
@code{CTRL-P} (previous) and @code{CTRL-N} (next) or the arrows on the
keyboard. Note that the history is not available on Macintosh systems.

The whole manual is available online by typing the command @code{help;} .
Explanation on single topics, e.g., on @code{intmat}, which defines a
matrix of integers, are obtained by

@smallexample
help intmat;
@end smallexample

@ifset singularmanual
@noindent This shows the text of @ref{intmat}, in the printed manual.
@end ifset
@ifclear singularmanual
@noindent This shows the text from node @code{intmat}, in the printed manual.
@end ifclear

Next, we define a
@tex
$3 \times 3$
@end tex
@ifinfo
3 x 3
@end ifinfo
 matrix of integers and initialize it with some values, row by row
from left to right:

@smallexample
intmat m[3][3] = 1,2,3,4,5,6,7,8,9;
@end smallexample

@noindent A single matrix entry may be selected and changed using
square brackets @code{[} and @code{]}.

@smallexample
m[1,2]=0;
m;
@expansion{} 1,0,3,
@expansion{} 4,5,6,
@expansion{} 7,8,9
@end smallexample

To calculate the trace of this matrix, we use a @code{for} loop. The
curly brackets @code{@{} and @code{@}} denote the beginning resp.@:
end of a block. If you define a variable without giving an initial
value, as the variable @code{tr} in the example below, @sc{Singular}
assigns a default value for the specific type. In this case, the default
value for integers is @code{0}. Note that the integer variable @code{j}
has already been defined above.

@smallexample
int tr;
for ( j=1; j <= 3; j++ ) @{ tr=tr + m[j,j]; @}
tr;
@expansion{} 15
@end smallexample

Variables of type string can also be defined and used without a ring
being active. Strings are delimited by @code{"} (double quotes). They
may be used to comment the output of a computation or to give it a nice
format. If a string contains valid @sc{Singular} commands, it can be
executed using the function @code{execute}. The result is the same as if
the commands would have been written on the command line. This feature
is especially useful to define new rings inside procedures.

@smallexample
"example for strings:";
@expansion{} example for strings:
string s="The element of m ";
s = s + "at position [2,3] is:";  // concatenation of strings by +
s , m[2,3] , ".";
@expansion{} The element of m at position [2,3] is: 6 .
s="m[2,1]=0; m;";
execute(s);
@expansion{} 1,0,3,
@expansion{} 0,5,6,
@expansion{} 7,8,9
@end smallexample

This example shows that expressions can be separated by @code{,} (comma)
giving a list of expressions. @sc{Singular} evaluates each expression in
this list and prints all results separated by spaces.

@c ------------------------------------------------------------------
@node Rings and standard bases, Procedures and libraries, First steps, Getting started
@ifset singularmanual
@subsection Rings and standard bases
@end ifset
@ifclear singularmanual
@section Rings and standard bases
@end ifclear
@cindex Rings and standard bases

To calculate with objects as ideals, matrices, modules, and polynomial
vectors, a ring has to be defined first.

@smallexample
ring r = 0,(x,y,z),dp;
@end smallexample

The definition of a ring consists of three parts: the first part
determines the ground field, the second part determines the names of the
ring variables, and the third part determines the monomial ordering to
be used. So the example above declares a polynomial ring called @code{r}
with a ground field of characteristic 
@ifinfo
@math{0}
@end ifinfo
@tex
$0$
@end tex
 (i.e., the rational
numbers) and ring variables called @code{x}, @code{y}, and @code{z}. The
@code{dp} at the end means that the degree reverse lexicographical
ordering should be used.

Other ring declarations:

@table @code
@item ring r1=32003,(x,y,z),dp;
characteristic 32003, variables @code{x}, @code{y}, and @code{z} and
ordering @code{dp}.

@item ring r2=32003,(a,b,c,d),lp;
characteristic 32003, variable names @code{a}, @code{b}, @code{c},
@code{d} and lexicographical ordering.

@item ring r3=7,(x(1..10)),ds;
characteristic 7, variable names @code{x(1)},@dots{},@code{x(10)}, negative
degree reverse lexicographical ordering (@code{ds}).

@item ring r4=(0,a),(mu,nu),lp;
transcendental extension of 
@ifinfo
@math{Q}
@end ifinfo
@tex
$Q$
@end tex
 by 
@ifinfo
@math{a}
@end ifinfo
@tex
$a$
@end tex
, variable names
@code{mu} and @code{nu}.

@item ring r5=real,(a,b),lp;
floating point numbers (single machine precision),
variable names @code{a} and @code{b}.

@item ring r6=(real,50),(a,b),lp;
floating point numbers with extended precision of 50 digits,
variable names @code{a} and @code{b}.

@item ring r7=(complex,50,i),(a,b),lp;
complex floating point numbers with extended precision of 50 digits
and imaginary unit @code{i},
variable names @code{a} and @code{b}.
@end table

@c Another valid characteristic would be, for example, a prime number less
@c or equal to 32003. The name of the ring variables may be any
@c valid @sc{Singular} name. Even indexed names are allowed, so
@c @code{x(1..10)} specifies the ring variables @code{x(1)}, @dots{},
@c @code{x(10)}. @sc{Singular} offers the possibility to calculate with any
@c monomial ordering, some orderings are predefined with special names like
@c @code{dp} in the example above. Another important example is the
@c lexicographical ordering called @code{lp}.
@c
Typing the name of a ring prints its definition. The example below
shows that the default ring in @sc{Singular} is 
@ifinfo
@math{Z/32003[x,y,z]}
@end ifinfo
@tex
$Z/32003[x,y,z]$
@end tex

with degree reverse lexicographical ordering:

@smallexample
@c computed example Rings_and_standard_bases start.doc:494 
ring r8;
r8;
@expansion{} //   characteristic : 32003
@expansion{} //   number of vars : 3
@expansion{} //        block   1 : ordering dp
@expansion{} //                  : names    x y z 
@expansion{} //        block   2 : ordering C
@c end example Rings_and_standard_bases start.doc:494
@end smallexample

Defining a ring makes this ring the current active basering, so each
ring definition above switches to a new basering. The concept of rings
in @sc{Singular} is discussed in detail in
@ifset singularmanual
@ref{Rings and orderings}.
@end ifset
@ifclear singularmanual
the chapter "Rings and orderings" of the @sc{Singular} manual.
@end ifclear

The basering is now @code{r8}. Since we want to calculate in the ring
@code{r}, which we defined first, we have to switch back to it. This can
be done using the function @code{setring}:

@smallexample
setring r;
@end smallexample

Once a ring is active, we can define polynomials. A monomial, say
@tex
$x^3$
@end tex
@ifinfo
x^3
@end ifinfo
may be entered in two ways: either using the power operator @code{^},
saying @code{x^3}, or in short-hand notation without operator, saying
@code{x3}. Note that the short-hand notation is forbidden if the name
of the ring variable consists of more than one character. Note, that
@sc{Singular} always expands brackets and automatically sorts the terms
with respect to the monomial ordering of the basering.

@smallexample
poly f =  x3+y3+(x-y)*x2y2+z2;
f;
@expansion{} x3y2-x2y3+x3+y3+z2
@end smallexample

The command @code{size} determines in general the number of ''single
entries`` in an object. In particular, for polynomials, @code{size}
determines the number of monomials.

@smallexample
size(f);
@expansion{} 5
@end smallexample

A natural question is to ask if a point, e.g., @code{(x,y,z)=(1,2,0)}, lies
on the variety defined by the polynomials @code{f} and @code{g}. For
this we define an ideal generated by both polynomials, substitute the
coordinates of the point for the ring variables, and check if the result
is zero:

@smallexample
poly g =  f^2 *(2x-y);
ideal I = f,g;
ideal J = subst(I,var(1),1);
J = subst(J,var(2),2);
J = subst(J,var(3),0);
J;
@expansion{} J[1]=5
@expansion{} J[2]=0
@end smallexample

@noindent Since the result is not zero, the point @code{(1,2,0)} does
not lie on the variety @code{V(f,g)}.

Another question is to decide whether some function vanishes on a
variety, or in algebraic terms if a polynomial is contained in a given
ideal. For this we calculate a standard basis using the command
@code{groebner} and afterwards reduce the polynomial with respect to
this standard basis.

@smallexample
ideal sI = groebner(f);
reduce(g,sI);
@expansion{} 0
@end smallexample

@noindent As the result is @code{0} the polynomial @code{g} belongs to the
ideal defined by @code{f}.

The function @code{groebner}, like many other functions in
@sc{Singular}, prints a protocol during calculations, if desired. The
command @code{option(prot);} enables protocolling whereas
@code{option(noprot);} turns it off.
@ifset singularmanual
@ref{option}, explains the meaning
of the different symbols printed during calculations.
@end ifset

The command @code{kbase} calculates a basis of the polynomial ring
modulo an ideal, if the quotient ring is finite dimensional.
As an example we calculate the Milnor number of a
hypersurface singularity in the global and local case. This is the
vector space dimension of the polynomial ring modulo the Jacobian ideal
in the global case resp.@: of the power series ring modulo the Jacobian
ideal in the local case. @xref{Critical points}, for a detailed
explanation.

The Jacobian ideal is obtained with the command @code{jacob}.

@smallexample
ideal J = jacob(f);
@expansion{} // ** redefining J **
J;
@expansion{} J[1]=3x2y2-2xy3+3x2
@expansion{} J[2]=2x3y-3x2y2+3y2
@expansion{} J[3]=2z
@end smallexample

@noindent @sc{Singular} prints the line @code{// ** redefining J
**}. This indicates that we have previously defined a variable with name
@code{J} of type ideal (see above).

To obtain a representing set of the quotient vector space we first
calculate a standard basis, then we apply the function @code{kbase} to
this standard basis.

@smallexample
J = groebner(J);
ideal K = kbase(J);
K;
@expansion{} K[1]=y4
@expansion{} K[2]=xy3
@expansion{} K[3]=y3
@expansion{} K[4]=xy2
@expansion{} K[5]=y2
@expansion{} K[6]=x2y
@expansion{} K[7]=xy
@expansion{} K[8]=y
@expansion{} K[9]=x3
@expansion{} K[10]=x2
@expansion{} K[11]=x
@expansion{} K[12]=1
@end smallexample

@noindent Then

@smallexample
size(K);
@expansion{} 12
@end smallexample

@noindent gives the desired vector space dimension
@tex
$K[x,y,z]/\hbox{\rm jacob}(f)$.
@end tex
@ifinfo
K[x,y,z]/jacob(f).
@end ifinfo
As in @sc{Singular} the functions may take the input directly from
earlier calculations, the whole sequence of commands may be written
in one single statement.

@smallexample
size(kbase(groebner(jacob(f))));
@expansion{} 12
@end smallexample

When we are not interested in a basis of the quotient vector space, but
only in the resulting dimension we may even use the command @code{vdim}
and write:

@smallexample
vdim(groebner(jacob(f)));
@expansion{} 12
@end smallexample

@c ------------------------------------------------------------------
@node Procedures and libraries, Change of rings, Rings and standard bases, Getting started
@ifset singularmanual
@subsection Procedures and libraries
@end ifset
@ifclear singularmanual
@section Procedures and libraries
@end ifclear
@cindex Procedures and libraries

@sc{Singular} offers a comfortable programming language, with a syntax
close to C. So it is possible to define procedures which collect several
commands to a new one. Procedures are defined with the keyword
@code{proc} followed by a name and an optional parameter list with
specified types.  Finally, a procedure may return values using the
command @code{return}.

Define the following procedure called @code{Milnor}:

@smallexample
proc Milnor (poly h)
@{
  return(vdim(groebner(jacob(h))));
@}
@end smallexample

Note: if you have entered the first line of the procedure and pressed
@code{RETURN}, @sc{Singular} prints the prompt @code{.} (dot) instead of
the usual prompt @code{>} . This shows that the input is incomplete and
@sc{Singular} expects more lines. After typing the closing curly
bracket, @sc{Singular} prints the usual prompt indicating that the input
is now complete.

@noindent Then call the procedure:

@smallexample
Milnor(f);
@expansion{} 12
@end smallexample

@noindent Note that the result may depend on the basering as we will
see in the next chapter.

The distribution of  @sc{Singular} contains  several libraries, each of
which is a collection of useful
procedures based on the kernel commands, which extend the functionality
of @sc{Singular}. The command @code{help "all.lib";} lists all libraries
together with a one-line explanation.
@c The command @code{help}
@c library_name@code{;} lists all procedures of the library, @code{help}
@c proc_name@code{;} shows an explanation of the procedure after the
@c library has been loaded. The command @code{LIB "all.lib";} loads all
@c libraries.

One of these libraries is @code{sing.lib} which already contains a
procedure called @code{milnor} to calculate the Milnor number not only
for hypersurfaces but more generally for complete intersection
singularities.

Libraries are loaded with the command @code{LIB}. Some additional
information during the process of loading is displayed on the screen,
which we omit here.

@smallexample
LIB "sing.lib";
@end smallexample

As all input in @sc{Singular} is case sensitive, there is no conflict with
the previously  defined procedure @code{Milnor}, but the result is the same.

@smallexample
milnor(f);
@expansion{} 12
@end smallexample

The procedures in a library have a help part
which is displayed by typing

@smallexample
help milnor;
@c @expansion{} // proc milnor from lib sing.lib
@c @expansion{} proc milnor (ideal i)
@c @expansion{} USAGE:   milnor(i); i ideal or poly
@c @expansion{} RETURN:  Milnor number of i, if i is ICIS (isolated complete intersection
@c @expansion{}          singularity) in generic form, resp. -1 if not
@c @expansion{} NOTE:    use proc nf_icis to put generators in generic form
@c @expansion{}          printlevel >=0: display comments (default)
@c @expansion{} EXAMPLE: example milnor; shows an example
@c @expansion{}
@end smallexample

@noindent as well as some examples, which are executed by

@smallexample
example milnor;
@c @expansion{} // proc milnor from lib sing.lib
@c @expansion{} EXAMPLE:
@c @expansion{}    int p      = printlevel;
@c @expansion{}    printlevel = 1;
@c @expansion{}    ring r     = 32003,(x,y,z),ds;
@c @expansion{}    ideal j    = x5+y6+z6,x2+2y2+3z2,xyz+yx;
@c @expansion{}    milnor(j);
@c @expansion{} //sequence of discriminant numbers: 100,149,70
@c @expansion{} 21
@c @expansion{}    poly f     = x7+y7+(x-y)^2*x2y2+z2;
@c @expansion{}    milnor(f);
@c @expansion{} 28
@c @expansion{}    printlevel = p;
@c @expansion{}
@end smallexample

@noindent Likewise, the library itself has a help part, to show a list of
all the functions
available for the user which are contained in the library.

@smallexample
help sing.lib;
@end smallexample

@noindent The output of the help commands is omitted here.

@c ------------------------------------------------------------------
@node Change of rings, Modules and their annihilator, Procedures and libraries, Getting started
@ifset singularmanual
@subsection Change of rings
@end ifset
@ifclear singularmanual
@section Change of rings
@end ifclear
@cindex Change of rings

To calculate the local Milnor number we have to do the calculation with the
same commands in a ring with local ordering.
@ifset singularmanual
Define the localization of the polynomial ring at the origin
(@pxref{Polynomial data}, and @ref{Mathematical background}).
@end ifset
@ifclear singularmanual
Define the localization of the polynomial ring at the origin.
@end ifclear

@smallexample
ring rl = 0,(x,y,z),ds;
@end smallexample

This ordering determines the standard basis which will be calculated.
Fetch the polynomial defined in the ring @code{r} into this new ring,
thus avoiding retyping the input.

@smallexample
poly f = fetch(r,f);
f;
@expansion{} z2+x3+y3+x3y2-x2y3
@end smallexample

@noindent Instead of @code{fetch} we can use the function @code{imap}
which is more general but less efficient.
@ifset singularmanual
The most general way to fetch data from one ring to another is to use maps,
this will be explained in @ref{map}.
@end ifset
@ifclear singularmanual
The most general way to fetch data from one ring to another is to use maps.
@end ifclear

In this ring the terms are ordered by increasing exponents. The local Milnor
number is now

@smallexample
Milnor(f);
@expansion{} 4
@end smallexample

This shows that @code{f} has outside the origin in affine 3-space
singularities with local Milnor number adding up to
@tex
$12-4=8$.
@end tex
@ifinfo
12-4=8.
@end ifinfo
Using global and local orderings as above is a convenient way to check
whether a variety has singularities outside the origin.

The command @code{jacob} applied twice gives the Hessian of @code{f}, a
3x3 - matrix.

@smallexample
matrix H = jacob(jacob(f));
H;
@expansion{} H[1,1]=6x+6xy2-2y3
@expansion{} H[1,2]=6x2y-6xy2
@expansion{} H[1,3]=0
@expansion{} H[2,1]=6x2y-6xy2
@expansion{} H[2,2]=6y+2x3-6x2y
@expansion{} H[2,3]=0
@expansion{} H[3,1]=0
@expansion{} H[3,2]=0
@expansion{} H[3,3]=2
@end smallexample

The @code{print} command displays the matrix in a nicer form.

@smallexample
print(H);
@expansion{} 6x+6xy2-2y3,6x2y-6xy2,  0,
@expansion{} 6x2y-6xy2,  6y+2x3-6x2y,0,
@expansion{} 0,          0,          2
@end smallexample

We may calculate the determinant and (the ideal generated by all) minors of
a given size.

@smallexample
det(H);
@expansion{} 72xy+24x4-72x3y+72xy3-24y4-48x4y2+64x3y3-48x2y4
minor(H,1);  // the 1x1 - minors
@expansion{} _[1]=2
@expansion{} _[2]=6y+2x3-6x2y
@expansion{} _[3]=6x2y-6xy2
@expansion{} _[4]=6x2y-6xy2
@expansion{} _[5]=6x+6xy2-2y3
@end smallexample

The algorithm of the standard basis computations may be
affected by the command @code{option}. For example, a reduced standard
basis of the ideal generated by the
@tex
$1 \times 1$-minors
@end tex
@ifinfo
1 x 1 - minors
@end ifinfo
 of H  is obtained in the following way:
@smallexample
option(redSB);
groebner(minor(H,1));
@expansion{} _[1]=1
@end smallexample

This shows that 1 is contained in the ideal of the
@tex
$1 \times 1$-minors,
@end tex
@ifinfo
1 x 1 - minors,
@end ifinfo
hence the corresponding variety is empty.
@c Coming back to some mathematical considerations, we study the problem how
@c to calculate some ....

@c ------------------------------------------------------------------
@c REMEMBER TO EDIT NEXT AND PREVIOUS NODE IF YOU UNCOMMENT THIS NODE!
@c @node Maps and elimination, Modules and their annihilator, Change of rings, Getting started
@c @subsection Maps and elimination
@c @cindex Maps and elimination

@c ------------------------------------------------------------------
@node Modules and their annihilator, Resolution, Change of rings, Getting started
@ifset singularmanual
@subsection Modules and their annihilator
@end ifset
@ifclear singularmanual
@section Modules and their annihilator
@end ifclear
@cindex Modules and and their annihilator

Now we shall give three more advanced examples.

@sc{Singular} is able to handle modules over all the rings,
which can be defined as a basering. A free module of rank @code{n}
is defined as follows:

@smallexample
ring rr;
int n = 4;
freemodule(4);
@expansion{} _[1]=gen(1)
@expansion{} _[2]=gen(2)
@expansion{} _[3]=gen(3)
@expansion{} _[4]=gen(4)
typeof(_);
@expansion{} module
print(freemodule(4));
@expansion{} 1,0,0,0,
@expansion{} 0,1,0,0,
@expansion{} 0,0,1,0,
@expansion{} 0,0,0,1
@end smallexample

To define a module, we give a list of vectors generating a submodule of
a free module. Then this set of vectors may be identified with the
columns of a matrix.  For that reason in @sc{Singular} matrices and
modules may be interchanged. However, the representation is different
(modules may be considered as sparse represented matrices).

@smallexample
ring r =0,(x,y,z),dp;
module MD = [x,0,x],[y,z,-y],[0,z,-2y];
matrix MM = MD;
print(MM);
@expansion{} x,y,0,
@expansion{} 0,z,z,
@expansion{} x,-y,-2y
@end smallexample

However the submodule 
@ifinfo
@math{MD}
@end ifinfo
@tex
$MD$
@end tex
 may also be considered as the module
of relations of the factor module
@tex
$r^3/MD$.
@end tex
@ifinfo
r^3/MD.
@end ifinfo
In this way, @sc{Singular} can treat arbitrary finitely generated modules
over the
@ifset singularmanual
basering (@pxref{Representation of mathematical objects}).
@end ifset
@ifclear singularmanual
basering.
@end ifclear

In order to get the module of relations of 
@ifinfo
@math{MD}
@end ifinfo
@tex
$MD$
@end tex
,
we use the command @code{syz}.

@smallexample
syz(MD);
@expansion{} _[1]=x*gen(3)-x*gen(2)+y*gen(1)
@end smallexample

We want to calculate, as an application, the annihilator of a given module.
Let
@tex
$M = r^3/U$,
@end tex
@ifinfo
M = r^3/U,
@end ifinfo
where U is our defining module of relations for the module
@tex
$M$.
@end tex
@ifinfo
M.
@end ifinfo

@smallexample
module U = [z3,xy2,x3],[yz2,1,xy5z+z3],[y2z,0,x3],[xyz+x2,y2,0],[xyz,x2y,1];
@end smallexample

Then, by definition, the annihilator of M is the ideal
@tex
$\hbox{ann}(M) = \{a \mid aM = 0 \}$
@end tex
@ifinfo
ann(M) = @{a | aM = 0 @}
@end ifinfo
which is by the description of M the same as
@tex
$\{ a \mid ar^3 \in U \}$.
@end tex
@ifinfo
@{ a | ar^3 contained in U@}.
@end ifinfo
Hence we have to calculate the quotient
@tex
$U \colon r^3 $.
@end tex
@ifinfo
U:r^3.
@end ifinfo
The rank of the free module is determined by the choice of U and is the
number of rows of the corresponding matrix. This may be determined by
the function @code{nrows}. All we have to do now is the following:

@smallexample
quotient(U,freemodule(nrows(U)));
@end smallexample

@noindent The result is too big to be shown here.

@c ------------------------------------------------------------------
@node Resolution,  , Modules and their annihilator, Getting started
@ifset singularmanual
@subsection Resolution
@end ifset
@ifclear singularmanual
@section Resolution
@end ifclear
@cindex Resolution

There are several commands in @sc{Singular} for computing free resolutions.
The most general command is @code{res(... ,n)} which determines heuristically
what method to use for the given problem. It computes the free resolution
up to the length 
@ifinfo
@math{n}
@end ifinfo
@tex
$n$
@end tex
, where 
@ifinfo
@math{n=0}
@end ifinfo
@tex
$n=0$
@end tex
 corresponds to the full resolution.

Here we use the possibility to inspect the calculation process using the
option @code{prot}.

@smallexample
ring R;      // the default ring in char 32003
R;
@expansion{} //   characteristic : 32003
@expansion{} //   number of vars : 3
@expansion{} //        block   1 : ordering dp
@expansion{} //                  : names    x y z
@expansion{} //        block   2 : ordering C
ideal I = x4+x3y+x2yz,x2y2+xy2z+y2z2,x2z2+2xz3,2x2z2+xyz2;
option(prot);
resolution rs = res(I,0);
@expansion{} using lres
@expansion{} 4(m0)4(m1).5(m1)g.g6(m1)...6(m2)..
@end smallexample

@noindent Disable this protocol with

@smallexample
option(noprot);
@end smallexample

When we enter the name of the calculated resolution, we get a pictorial
description of the minimized resolution where the exponents denote the rank of the
free modules. Note that the calculated resolution itself may not yet be minimal.

@smallexample
rs;
@expansion{} 1      4      5      2      0
@expansion{}R  <-- R  <-- R  <-- R  <-- R
@expansion{}
@expansion{}0      1      2      3      4
print(betti(rs),"betti");
@expansion{}            0     1     2     3
@expansion{} ------------------------------
@expansion{}     0:     1     -     -     -
@expansion{}     1:     -     -     -     -
@expansion{}     2:     -     -     -     -
@expansion{}     3:     -     4     1     -
@expansion{}     4:     -     -     1     -
@expansion{}     5:     -     -     3     2
@expansion{} ------------------------------
@expansion{} total:     1     4     5     2
@end smallexample

In order to minimize the resolution, that is to calculate the maps of the minimal 
free resolution, we use the command @code{minres}:

@smallexample
rs=minres(rs);
@end smallexample

A single module in this resolution is obtained (as usual) with the
brackets @code{[} and @code{]}. The @code{print} command can be used to
display a module in a more readable format:

@smallexample
print(rs[3]);
@expansion{} z3,   -xyz-y2z-4xz2+16z3,
@expansion{} 0,    -y2,
@expansion{} -y+4z,48z, 
@expansion{} x+2z, 48z, 
@expansion{} 0,    x+y-z  
@end smallexample

In this case, the output is to be interpreted as follows: the 3rd syzygy
module of R/I, @code{rs[3]}, is the rank-2-submodule of
@tex
$R^5$
@end tex
@ifinfo
R^5
@end ifinfo
generated by the vectors
@tex
$(z^3,0,-y+4z,x+2z,0)$ and $(-xyz-y^2z-4xz^2+16z^3,-y^2,48z,48z,x+y-z)$.
@end tex
@ifinfo
(z^3,0,-y+4z,x+2z,0) and (-xyz-y^2z-4xz^2+16z^3,-y^2,48z,48z,x+y-z).
@end ifinfo

