% Id: singcard.tex,v 1.1 2003/08/08 14:27:06 pertusus Exp $

%
% singcard.tex - common macros for Singular reference cards.
%
% Some of the macros/settings ripped from gdb reference card.
%

%
% - declare configurable parameters.
%
\newdimen\totalwidth
\newdimen\totalheight
\newdimen\hmargin
\newdimen\vmargin

\newdimen\barwidth
\newdimen\barheight

%%% START CONFIGURABLE SECTION

%
% - configure your papersize.
%

% US letter paper (8.5x11in)
%\totalwidth=11in		% total width of paper
%\totalheight=8.5in		% total height of paper
%\hmargin=.25in			% horizontal margin width
%\vmargin=.25in			% vertical margin width

% A4 paper
\totalwidth=29.7cm
\totalheight=21cm
\hmargin=6mm
\vmargin=6mm

%
% - configure your folding guides.
%
\barwidth=.1pt			% width of the folding guide
\barheight=2pt			% height of the folding guide

% For NO printed folding guide,
% comment out other \def\vdecor's and uncomment:
%\def\vdecor{\hskip\hmargin plus1fil\hskip\barwidth plus1fil\hskip\hmargin plus1fil}

% For SOLID LINE folding guide,
% comment out other \def\vdecor's and uncomment:
%\def\vdecor{\hskip\hmargin plus1fil \vrule width \barwidth \hskip\hmargin plus1fil}

% For SMALL MARKS NEAR TOP AND BOTTOM as folding guide,
% comment out other \def\vdecor's and uncomment:
\def\vdecor{\hskip\hmargin plus1fil
\vbox to \vsize{\hbox to \barwidth{\vrule height\barheight width\barwidth}\vfill
\hbox to \barwidth{\vrule height\barheight width\barwidth}}%THIS PERCENT SIGN IS ESSENTIAL
\hskip\hmargin plus1fil}

%%% END CONFIGURABLE SECTION

%
% - compute parameters of three column format.
%
\newdimen\fullhsize		% width of area without margins
\newdimen\itemwidth		% width of item column
\newdimen\definitionwidth	% width of definition column
\newdimen\temp			% only for temporary use

% adjust the offsets so the margins are measured *from paper edge*
\hoffset=-1in \advance \hoffset by \hmargin
\voffset=-1in \advance \voffset by \vmargin

% fullhsize = totalwidth - (2 * hmargin)
\fullhsize=\totalwidth
\temp=\hmargin \multiply \temp by 2 \advance \fullhsize by -\temp

% hsize = (fullhsize - (4 * hmargin) - (2 * barwidth)) / 3
\hsize=\fullhsize
\temp=\hmargin \multiply \temp by 4 \advance \hsize by -\temp
\temp=\barwidth \multiply \temp by 2 \advance \hsize by -\temp
\divide \hsize by 3

% vsize = totalheight - (2 * vmargin)
\vsize=\totalheight
\temp=\vmargin \multiply \temp by 2 \advance \vsize by -\temp

% default itemwidth = hsize * 1/4
% default definitionwidth = hsize * 3/4
\temp=\hsize \divide \temp by 4
\itemwidth=\temp
\definitionwidth=\hsize \advance \definitionwidth by -\itemwidth

%
% - miscellaneous parameters.
%
\tolerance=2000
\hyphenpenalty=500

\parindent=0pt
\parskip=0pt
\baselineskip=9pt

\footline={\vbox to0pt{\hss}}

%
% - font definitions.
%
\font\rm=cmr7
\font\bf=cmbx7
\font\it=cmti7
\font\bit=cmbxti10 at 7pt
\font\tt=cmtt8			% cmtt7 seems too small
\font\sc=cmcsc10 at 7pt

% huge fonts
\font\hbf=cmbx12

% large fonts
\font\lbf=cmbx10

% small fonts
\font\srm=cmr6
\font\ssc=cmcsc10 at 6pt

% tiny fonts
\font\tit=cmti7 at 5pt

% math fonts
\font\mathrm=cmr7
\font\smathrm=cmr5
\font\mathit=cmmi7
\font\smathit=cmmi5
\font\mathsy=cmsy7
\font\smathsy=cmsy5
\font\mathex=cmex7

\textfont0=\mathrm \scriptfont0=\smathrm
\textfont1=\mathit \scriptfont1=\smathit
\textfont2=\mathsy \scriptfont2=\smathsy
\textfont3=\mathex \scriptfont3=\mathex

%
% - macros for tree column format.
%
\def\fulline{\hbox to \fullhsize}
\def\columnbox{\leftline{\pagebody}}
\def\tripleformat{\shipout
		  \vbox{\fulline{\box\leftcolumn\vdecor
				 \box\centercolumn\vdecor
				 \columnbox}}%
		  \advancepageno}
\def\bye{\par\vfill
	 \supereject
	 \if R\currentcolumn\null\vfill\eject\fi
	 \end}

%
% - miscellaneous macros.
%
% \Singular: for sake of corporate identity
% \ctl: print control keys
% \opt: print optional arguments
% \alt: separate alternative arguments
% \rep: denote repitition of arguments
% \rmslash: print a slash in roman type face
% \eg, \ie: obvious
% \verysmallskip: skip a very small amount (used between \sec entries).
%   Extra skip does not harm in the applications of \verysmallskip,
%   so we choose a large stretch for it.
% \strut: redefinition of TeX's \strut macro
%
\def\Singular{{\sc Singular}}
\def\ctl#1{{\tt C-#1}}
\def\opt#1{{\rm[{\tt #1}]}}
\def\alt{\hbox{\hskip 1pt$\mid$\hskip 1pt}}
\def\rep{$\ldots$}
\def\rmslash{{\rm /}}
\def\eg{e.g.~}
\def\ie{i.e.~}

\newskip\verysmallskipamount
\verysmallskipamount=1pt plus 0.333333pt minus 0.333333pt
\def\verysmallskip{\vskip \verysmallskipamount}

\newbox\sevenptstrutbox
\setbox\sevenptstrutbox=\hbox{\vrule height6pt depth3pt width0pt}
\def\strut{\relax\ifmmode\copy\sevenptstrutbox\else\unhcopy\sevenptstrutbox\fi}

%
% \sec - insert a new section into the card.
%
% First argument is section title, second argument the item column
% width (defaults to 1/4*hsize), third the section itself.
%
% \setitemwidth: auxiliary macro
%
\def\setitemwidth#1{\itemwidth=#1
		    \definitionwidth=\hsize
		    \advance\definitionwidth by -\itemwidth}
\long\def\sec#1(#2)#3\endsec{{%
	\def\\{#2}\ifx\empty\\ \else\setitemwidth{#2}\fi
	\halign{% column 1
		\vtop{\hsize=\itemwidth\strut\tt##\strut}&
		% column 2
		\vtop{\hsize=\definitionwidth\strut\rm##\strut}\cr
		% tail of \long\def fills in halign body with argument
		\noalign{\bigskip\lbf #1\verysmallskip}%
		#3}}}

%
% - macros to use inside a \sec.
%
% \subsec: insert a sub heading into a \sec
% \sectext: insert non-aligned text into a \sec
% \longentry: insert an entry with a long item field into a \sec
% \longitem: insert an entry with a long item field and no
%   definition field into a \sec
% \entryskip: insert glue between two entries
%
\def\subsec#1{\noalign{\smallskip\bf #1}}
\long\def\sectext#1\cr{\noalign{\strut\hyphenpenalty=50 #1\strut}}
% the `\ ' forces a non empty item field
\long\def\longentry#1&#2\cr{\noalign{\strut\tt #1\strut}\ &#2\cr}
\long\def\longitem#1\cr{\noalign{\strut\tt #1\strut}}

\newskip\entryskipamount
\entryskipamount=\smallskipamount
\def\entryskip{\noalign{\vskip \entryskipamount}}

%
% - initialization.
%

% set output routine
\newbox\leftcolumn
\newbox\centercolumn
\let\currentcolumn=L
\output={\if L\currentcolumn
	    \global\setbox\leftcolumn=\columnbox \global\let\currentcolumn=C%
	 \else
	    \if C\currentcolumn
	       \global\setbox\centercolumn=\columnbox \global\let\currentcolumn=R%
	    \else
	       \tripleformat \global\let\currentcolumn=L%
	    \fi
	 \fi}

% switch to our default font
\rm

% ragged format
\raggedbottom
\raggedright
